\documentclass[a4paper]{article}
\usepackage[titletoc]{appendix}
\usepackage[margin=1in]{geometry}

\usepackage[T1]{fontenc}
\usepackage[adobe-utopia]{mathdesign}			

\usepackage[framemethod=tikz]{mdframed}
\usepackage{lipsum}


\usepackage[utf8]{inputenc}
 
\usepackage{listings}
%\usepackage{color}


\input{listings-stata.tex}
 
\definecolor{codegreen}{rgb}{0,0.6,0}
\definecolor{codegray}{rgb}{0.5,0.5,0.5}
\definecolor{codepurple}{rgb}{0.58,0,0.82}
\definecolor{backcolour}{rgb}{0.95,0.95,0.92}
 
\lstdefinestyle{mystyle}{
    backgroundcolor=\color{backcolour},   
    commentstyle=\color{codegreen},
    keywordstyle=\color{magenta},
    numberstyle=\tiny\color{codegray},
    stringstyle=\color{codepurple},
    breakatwhitespace=false,         
    breaklines=true,                 
    captionpos=b,                    
    keepspaces=false,                                  
    numbersep=5pt,                  
    showspaces=false,                
    showstringspaces=false,
    showtabs=false,                  
    basicstyle= \footnotesize\ttfamily,
    tabsize = 4
}
 
\lstset{style=mystyle}




\begin{document}

\title{Stata Code Highlight Version - Chapter \#3}
\author{ }
\date{\today}
\maketitle



\section*{Question 6}
\begin{lstlisting}[language=stata]
. // download data from: http://hdl.handle.net/10079/6hdr852
. // copy and paste the url to your web browser
. 
. import delim "Clingingsmith_et_al_QJE_2009dta.csv",clear
(8 vars, 958 obs)

. set seed 1234567
. rename success D
. rename views Y
//findit tsrtest
. //package name:  st0158.pkg install
. 
. cap program drop ate
. program define ate, rclass
  1.         args Y D
  2.     sum `Y' if `D'==1, meanonly
  3.     local Y_treat=r(mean)
  4.     sum `Y' if `D'==0, meanonly
  5.     local Y_con=r(mean)
  6.     return scalar ate_avg = `Y_treat'-`Y_con'
  7. end

. // ssc install tsrtest
. tsrtest D r(ate_avg) using 3_6_resam.dta, overwrite: ate Y D
Two-sample randomization test for theta=r(ate_avg) of ate Y D by D

Combinations:   8.4503047638e+285 = (958 choose 448)
Assuming null=0
Observed theta: .4748

Minimum time needed for exact test (h:m:s):  4.2e+278:00:00
Reverting to Monte Carlo simulation.
Mode: simulation (10000 repetitions)

progress: |........................................|

 p=0.00190 [one-tailed test of Ho:  theta(D==0)<=theta(D==1)]
 p=0.99830 [one-tailed test of Ho:  theta(D==0)>=theta(D==1)]
 p=0.00360 [two-tailed test of Ho:  theta(D==0)==theta(D==1)]

Saving log file to 3_6_resam.dta...done.
.   
. preserve 
. use "3_6_resam.dta", clear

. global ate = theta[1]

. di $ate
.4748337

. drop if _n==1
(1 observation deleted)

. count if theta >= $ate
  19

. scalar p_onesided = r(N)/_N

. count if abs(theta) >= $ate
  36

. scalar p_twosided = r(N)/_N

. di "p.value.onesided = "p_onesided
p.value.onesided = .0019

. di "p.value.twosided = "p_twosided 
p.value.twosided = .0036

. restore
\end{lstlisting}



\section*{Question 7}
\begin{lstlisting}[language=stata]
. clear
. set seed 1234567
. set obs 10
number of observations (_N) was 0, now 10
. 
. input D Y

             D          Y
  1. 0 1 
  2. 0 0 
  3. 0 0 
  4. 0 4 
  5. 0 3 
  6. 1 2 
  7. 1 11 
  8. 1 14 
  9. 1 0 
 10. 1 3 
. 
. gen Y_star= Y+D*(-7)
. 
. cap program drop ate

. program define ate, rclass
  1.         args Y D
  2.         sum `Y' if `D'==1, meanonly
  3.         local Y_treat=r(mean)
  4.         sum `Y' if `D'==0, meanonly
  5.         local Y_con=r(mean)
  6.         return scalar ate_avg = `Y_treat'-`Y_con'
  7. end
. 
. // findit tsrtest (to install the package)
. tsrtest D r(ate_avg): ate Y_star D
Two-sample randomization test for theta=r(ate_avg) of ate Y_star D by D

Combinations:   252 = (10 choose 5)
Assuming null=0
Observed theta: -2.6

Minimum time needed for exact test (h:m:s):  0:00:00
Mode: exact

progress: |........................................|

 p=0.83730 [one-tailed test of Ho:  theta(D==0)<=theta(D==1)]
 p=0.20635 [one-tailed test of Ho:  theta(D==0)>=theta(D==1)]
 p=0.41270 [two-tailed test of Ho:  theta(D==0)==theta(D==1)]
. 
. // ate
. di r(obsvStat)       
-2.6

. 
. // p.value.onesided
. di r(lowertail)   
.20634921

\end{lstlisting}

\section*{Question 8}
\subsection*{part(a)}
\begin{lstlisting}[language=stata]
 // download data from : http://hdl.handle.net/10079/s1rn910
. // copy and paste the url to your web browser
. use "Titiunik_WorkingPaper_2010.csv.dta",clear 
. 
. set seed 1234567

.         rename term2year D
.         rename bills_introduced Y
.         rename texas0_arkansas1 block         
.         qui tabstat Y if block ==0, by(D) stat(mean) save       
.         scalar ate_texas = el(r(Stat2),1,1) - el(r(Stat1),1,1)         
.         qui tabstat Y if block ==1, by(D) stat(mean) save       
.         scalar ate_ark = el(r(Stat2),1,1) - el(r(Stat1),1,1)
.         
.         di "ate_texas="%18.5f ate_texas 
ate_texas=         -16.74167

.         di "ate_arkansas="%18.5f ate_ark        
ate_arkansas=         -10.09477

\end{lstlisting}
\subsection*{part(b)}
\begin{lstlisting}[language=stata]

 qui tabstat Y if block ==0, by(D) stat(v n) save        
. scalar se_texas = sqrt(el(r(Stat2),1,1)/el(r(Stat2),2,1) + /// 
>                                         el(r(Stat1),1,1)/el(r(Stat1),2,1))
.                                         
. 
. qui tabstat Y if block ==1, by(D) stat(v n) save        

. 
. scalar se_arkansas = sqrt(el(r(Stat2),1,1)/el(r(Stat2),2,1) + /// 
>                                         el(r(Stat1),1,1)/el(r(Stat1),2,1)) 
. 
. di "se_texas="%18.6f se_texas
se_texas=          9.345871

. di "se_arkansas="%18.6f se_arkansas
se_arkansas=          3.395979

\end{lstlisting}
\subsection*{part(c)}
\begin{lstlisting}[language=stata]
 qui tabstat Y, by(block) stat(n) save   
. 
. scalar ate_overall = el(r(Stat1),1,1)/_N*ate_texas + /// 
>                                          el(r(Stat2),1,1)/_N*ate_ark
. 
. 
. di %18.4f ate_overall
          -13.2168
. 
. // same as
. // teffects nnmatch (bills_introduced) (term2year), ematch(texas0_arkansas1)

\end{lstlisting}
\subsection*{part(e)}
\begin{lstlisting}[language=stata]
. scalar se_overall = sqrt((el(r(Stat1),1,1)/_N)^2*se_texas^2 + /// 
>                                          (el(r(Stat2),1,1)/_N)^2*se_arkansas^2)
.                                          
. di %18.5f se_overall
           4.74478

\end{lstlisting}
\subsection*{part(f)}
\begin{lstlisting}[language=stata]
 // calculate probs under block assignment
. bysort block: egen probs=mean(D). 
. 
. 
. cap program drop ate_block

. 
. program define ate_block, rclass
  1. args Y D probs
  2. tempvar ipw
  3. gen `ipw' = .
  4. // calculate inverse probability weight under block assignment
. replace `ipw' = `D'/`probs' + (1-`D')/(1-`probs')
  5. qui reg `Y' `D' [iw=`ipw']
  6. return scalar ate=_b[`D']
  7. end 
. 
. // ssc install ritest (to install ritest package)
. 
. //
. ritest D r(ate), strata(block) reps(10000) nodots: ///
> ate_block Y D probs
(66 missing values generated)
(66 real changes made)

      command:  ate_block Y D probs
        _pm_1:  r(ate)
  res. var(s):  D
   Resampling:  Permuting D
Clust. var(s):  __000000
     Clusters:  66
Strata var(s):  block
       Strata:  2

------------------------------------------------------------------------------
T            |     T(obs)       c       n   p=c/n   SE(p) [95% Conf. Interval]
-------------+----------------------------------------------------------------
       _pm_1 |   -13.2168      65   10000  0.0065  0.0008    .00502   .0082774
------------------------------------------------------------------------------
Note: Confidence interval is with respect to p=c/n.
Note: c = #{|T| >= |T(obs)|}

. // ate
. di el(r(b),1,1)
-13.216796

. 
. // p.value.twosided
. di el(r(p),1,1)
.0065
\end{lstlisting}


\section*{Question 9}
\subsection*{part(b)}

\begin{lstlisting}[language=stata]
 // download data from : http://hdl.handle.net/10079/1g1jx43
. // copy and paste the url to your web browser
. 
. use "Camerer_JPEsubset_1998.dta.dta", clear 
. 
. set seed 1234567

.         rename treatment D

.         rename pair block

.         rename preexperimentbets covs
. 
.         // calculate probs under block assignment
.         bysort block: egen probs=mean(D)

.         
.                 
.         // permuation to calculate F stat and one-side P value
.         ritest D e(F), strata(block) reps(10000) right nodots: ///
>         regress D covs

      Source |       SS           df       MS      Number of obs   =        34
-------------+----------------------------------   F(1, 32)        =      0.02
       Model |  .005024372         1  .005024372   Prob > F        =    0.8914
    Residual |  8.49497563        32  .265467988   R-squared       =    0.0006
-------------+----------------------------------   Adj R-squared   =   -0.0306
       Total |         8.5        33  .257575758   Root MSE        =    .51524

------------------------------------------------------------------------------
           D |      Coef.   Std. Err.      t    P>|t|     [95% Conf. Interval]
-------------+----------------------------------------------------------------
        covs |  -.0000386   .0002809    -0.14   0.891    -.0006109    .0005336
       _cons |   .5137818   .1335793     3.85   0.001     .2416896     .785874
------------------------------------------------------------------------------

      command:  regress D covs
        _pm_1:  e(F)
  res. var(s):  D
   Resampling:  Permuting D
Clust. var(s):  __000000
     Clusters:  34
Strata var(s):  block
       Strata:  17

------------------------------------------------------------------------------
T            |     T(obs)       c       n   p=c/n   SE(p) [95% Conf. Interval]
-------------+----------------------------------------------------------------
       _pm_1 |   .0189265    3736   10000  0.3736  0.0048  .3641064   .3831672
------------------------------------------------------------------------------
Note: Confidence interval is with respect to p=c/n.
Note: c = #{T >= T(obs)}

. 
.         // p.value
.         di el(r(p),1,1)
.3736



\end{lstlisting}

\subsection*{part(c)}
\begin{lstlisting}[language=stata]
. rename experimentbets change

. 
. tabstat change, by(D) stat(mean) save   

Summary for variables: change
     by categories of: D 

       D |      mean
---------+----------
       0 |  571.4118
       1 |  461.2353
---------+----------
   Total |  516.3235
--------------------

. 
. di "ATE ="%180.4f el(r(Stat2),1,1)-el(r(Stat1),1,1)
ATE =                 -110.1765


\end{lstlisting}
\subsection*{part(d)}
\begin{lstlisting}[language=stata]
 bysort block (D): gen pair_diff = change - change[_n+1]
(17 missing values generated)

. mean(pair_diff)

Mean estimation                   Number of obs   =         17

--------------------------------------------------------------
             |       Mean   Std. Err.     [95% Conf. Interval]
-------------+------------------------------------------------
   pair_diff |   110.1765   104.8377     -112.0695    332.4225
--------------------------------------------------------------

. 
. // the same as
. // teffects nnmatch (experimentbets block) (D)

\end{lstlisting}
\subsection*{part(e)}
\begin{lstlisting}[language=stata]

. cap program drop ate_block

. 
. program define ate_block, rclass
  1. args Y D probs
  2. tempvar ipw
  3. gen `ipw' = .
  4. // calculate inverse probability weight under block assignment
. replace `ipw' = `D'/`probs' + (1-`D')/(1-`probs')
  5. qui reg `Y' `D' [iw=`ipw']
  6. return scalar ate=_b[`D']
  7. end 

. 
. 
. ritest D r(ate), strata(block) reps(10000) nodots: ///
> ate_block change D probs
(34 missing values generated)
(34 real changes made)

      command:  ate_block change D probs
        _pm_1:  r(ate)
  res. var(s):  D
   Resampling:  Permuting D
Clust. var(s):  __000000
     Clusters:  34
Strata var(s):  block
       Strata:  17

------------------------------------------------------------------------------
T            |     T(obs)       c       n   p=c/n   SE(p) [95% Conf. Interval]
-------------+----------------------------------------------------------------
       _pm_1 |  -110.1765    3170   10000  0.3170  0.0047  .3078845   .3262222
------------------------------------------------------------------------------
Note: Confidence interval is with respect to p=c/n.
Note: c = #{|T| >= |T(obs)|}. 
. 
. // ate
. di el(r(b),1,1)
-110.17647

. 
. // p.value.twosided
. di el(r(p),1,1)
.317
\end{lstlisting}

\end{document}
