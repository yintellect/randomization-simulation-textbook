% %%%%%%%%%%%%%%%%%%%%%%%%%%%%%%%%%%%%%%%%%%%%%%%%%%%%%%%%%%%%%%%%%%%%%%%%%%%%%%%%%%%%%%%%%%%%
% PROBLEM SET LATEX TEMPLATE FILE
% DEFINE DOCUMENT STYLE, LOAD PACKAGES
\documentclass[11pt,notitlepage]{article}		% ADD COMMENTS USING A PERCENT SIGN
\usepackage{calc}
\usepackage{amsfonts}
\usepackage{amsthm}
\usepackage{amsmath, booktabs}
\usepackage{mathtools}
\usepackage{amssymb}
\usepackage{subfig}
\usepackage[none]{hyphenat} 
\usepackage{setspace}
\usepackage{fullpage}
\usepackage{verbatim}
\usepackage{graphicx}
\usepackage{tabularx}
\usepackage{longtable}
\usepackage{multicol}
\usepackage{multirow}
%\setlength{\parindent}{0in}		% uncomment to remove indent at start of paragraphs
\usepackage{pdflscape}
\usepackage[english]{babel}
\usepackage[pdftex]{hyperref}
\usepackage{natbib}
\usepackage{caption}
\usepackage{amsmath}
\usepackage{amsfonts}
\usepackage{graphics}
\usepackage{multirow}
\usepackage{graphics}
\usepackage{hyperref}
\usepackage{longtable}
\usepackage{latexsym}
\usepackage{rotating}
\usepackage{setspace}
\usepackage{layouts} 
\usepackage[titletoc]{appendix}
\DeclareGraphicsExtensions{.pdf,.jpg,.png}
\usepackage[margin=1in]{geometry}
\usepackage{enumerate}
\usepackage{float}


% FONTS
\usepackage[T1]{fontenc}					% always use this no matter what

% uncomment any one of these to see what it does to your font!
%\usepackage{pxfonts}
%\usepackage{cmbright}
%\usepackage{txfonts}
%\usepackage[adobe-utopia]{mathdesign}
%\usepackage{kpfonts}
%\usepackage{lmodern}
%\usepackage{newtxtext,newtxmath}



% DEFINE WHAT GOES INTO YOUR TITLE BEFORE THE DOCUMENT BEGINS
\title{POLS 4368 - Fall 2014 Midterm}
\author{Professor Donald P. Green \\
TA Alex Coppock}
\date{\today}

% %%%%%%%%%%%%%%%%%%%%%%%%%%%%%%%%%%%%%%%%%%%%%%%%%%%%%%%%%%%%%%%%%%%%%%%%%%%%%%%%%%%%%%%%%%%%
\begin{document}

\maketitle

This exam reviews concepts from FEDAI Chapters 1-6.  At the conclusion of this exam, please turn in this sheet along with your blue exam booklet.
\section{Section I}
Briefly define and state the significance of the following terms or phrases.  Use formal notation to make your definitions as clear as possible.
\begin{enumerate}
\item Randomization inference
\item Covariate adjustment
\item Clustered random assignment
\item The assumption of monotonicity in the context of two-sided noncompliance
\end{enumerate}

\section{Section II}
The following table was presented in Chapter 5. The results refer to the New Haven voter mobilization experiment, in which a random subset of the subject pool was assigned to be canvassed, but only some of those assigned to be canvassed were actually canvassed.  The outcome is voter turnout.


\begin{table}

    \begin{tabular}{lll}
    ~                                                    & Treatment Group & Control Group \\
    Turnout rate among those contacted by canvassers     & 54.43 (395)     & ~             \\
    Turnout rate among those not contacted by canvassers & 36.48 (1,050)   & 37.54 (5,645) \\
    Overall turnout rate                                 & 41.38 (1,445)   & 37.54 (5,645) \\
    \end{tabular}
\end{table}

\end{document}

